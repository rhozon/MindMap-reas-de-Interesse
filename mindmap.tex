\documentclass{article}
\usepackage[utf8]{inputenc}
\usepackage{lscape}
\usepackage{hyperref}
\hypersetup{
    colorlinks=true,
    linkcolor=blue,
    filecolor=magenta,      
    urlcolor=cyan,
}
\usepackage{tikz}
\usepackage[a4paper,margin={1.2in,1.5in},vmargin={1.2in,1.5in}]{geometry}
\geometry{paperwidth=210mm,paperheight=297mm,
textwidth=150mm,textheight=210mm,
top=23mm,bottom=23mm,
left=23mm,right=23mm}
\usepackage[colorlinks,linkcolor=blue,hyperindex]{hyperref}
\usepackage[brazil]{babel}
\usepackage{graphicx,color}
\usepackage{multicol}
\usetikzlibrary{mindmap}
\pagestyle{empty}
\usepackage{fancyhdr}
\pagestyle{fancy}
\fancyhead[LE,RO]{\includegraphics[height=39pt]{me}}
\lhead{\thepage}
\fancyfoot[]{}}
\renewcommand{\headrulewidth}{0.1pt}
%=============================================================


%=============================================================
\begin{document}
%=============================================================
.

\vspace{.5cm}


\begin{center}
\textbf{RODRIGO HERMONT OZON}
\end{center}

\begin{flushleft}
\textit{Documento para candidatura a vaga\\ de aluno Curso de Especialização em Data Science \& Big Data\footnote{Conforme recomendações dispostas no website http://dsbd.leg.ufpr.br/inscricoes/}}
\end{flushleft}
\rule[6cm]{6in}{0.1pt} 

\vspace{-5.5cm}
\begin{itemize} 
\item \textit{\textcolor{blue}{\href{mailto:rodrigoozon@yahoo.com.br}{Me envie e-mail:}}} ou me zap \textit{\textcolor{blue}{\href{https://api.whatsapp.com/send?phone=5541988382904&text=Ol\%C3\%A1\%20Rodrigo\%2C\%20vi\%20seu\%20portf\%C3\%B3lio\%20no\%20linkedin\%2C\%20tenho\%20interesse\%20em\%20conhecer\%20um\%20pouco\%20mais\%20do\%20seu\%20trabalho.\%20Podemos\%20conversar\%20a\%20respeito\%20\%3F}{+55 41 98838-2904}}} 
\item LinkeDin: \url{https://www.linkedin.com/in/rodrigohermontozon}

\item \href{http://lattes.cnpq.br/3532649625879285}{Consulte meu CV na Plataforma Lattes do CNPq}

\item No GitHub: \href{https://github.com/rhozon/Portfolio/blob/master/Portfolio.tex}{https://github.com/rhozon (aqui está disponível este portfolio na linguagem\\ \LaTeX utilizando o pacote Beamer)} \href{https://github.com/rhozon/Candidatura-UFPR}{e para este arquivo .tex aqui.}

\item Na galeria do Tableau Public: \href{https://public.tableau.com/profile/rodrigo.h.ozon#!/}{https://public.tableau.com/profile/rodrigo.h.ozon#!/}

\item Perfil no DrivenData \href{https://www.drivendata.org/users/rhozon/}{https://www.drivendata.org/users/rhozon/}
\end{itemize}
\rule[6cm]{6in}{0.1pt} 


\vspace{-5cm}

\begin{center}
\textbf{Carta de Intenções}
\end{center}

\begin{quote}
\textit{Necessito cursar a especialização pois careço muito aperfeiçoar e conhecer mais sobre as técnicas computacionais e teóricas a respeito de dados, coisa que minha formação acadêmica sempre foi deficiente. Como trabalho com dados econômicos fazem aproximadamente 12 anos, me interesse muito por métodos estatísticos contemplando as áreas de interesse relacionadas a estatística espacial e modelagem estatística. Também me interesso por estatística computacional e data storytelling quando emprego alguns modelos econométricos. }

\textit{Apesar da minha experiência com estatística florescer desde o final da minha graduação em economia (tenho um projeto no LEG até hoje) considero minha habilidade como cientista de dados muito limitada por não me considerar um programador nato para trabalhar com mais eficiência com as informações que deparo, porém minha paixão pela ciência da estatística sempre fez com que meu trabalho como economista fosse muito reconhecido por compreender qual melhor modelo econométrico aplico para determinada realidade econômica que tratei. Tenho trabalhado recentemente com dashboards e ferramentas de BI, porém minhas habilidades com bancos de dados me parecem distantes das necessidades ainda.}

\textit{Como tenho mestrado em desenvolvimento econômico pela UFPR, considero relevante o conhecimento teórico que me fora iniciado para a melhoria de processos de reflexão, modelagem e melhoria de tomada de decisão para cada problema econômico estudado. Na minha experiência relato o projeto que coordenei onde utilizei análise multivariada para construir um indicador de avaliação de ambiente de negócios para as MPEs. No mesmo ensejo fui professor universitário nos níveis de graduação e pós e sempre fui um apaixonado pelo conhecimento acadêmico/científico.}

\textit{Em relação ao curso espero conseguir aprender a lidar, organizar, apresentar e analisar grandes e diferentes tipos de dados com os métodos estatísticos e computacionais adequados e corrigir algumas de minhas falhas de formação no âmbito tecnológico e com esta mentalidade tenho plena certeza de que este curso irá proporcionar novos níveis em minha carreira profissional.}
\end{quote}
    

\newpage

.
\begin{center}
\textbf{Experiência Profissional}
\end{center}

\rule[6cm]{6in}{0.1pt} 
\vspace{-5cm}
%====================================================
\begin{flushleft}

\fbox{\textbf{\textcolor{magenta}{\href{https://www.kietec.com.br/}{KIE-TEC - KNOWLEDGE, INNOVATION & EXCELLENCE IN TECHNOLOGY}}}}

\vspace{.25cm}
\textbf{\textit{Consultor de Business Intelligence}}

[Curitiba desde Nov de 2019 atualmente]
\end{flushleft}

\begin{itemize}
\item Atuação na área de levantamento de requisitos, elaboração e sugestão de implantação de modelos preditivos de decisão, análise de dados e seus devidos cruzamentos.
\item Acompanhamento de outros consultores \textit{in loco} nos trabalhos de ETLs e construção de painéis no Tableau
\end{itemize}
%====================================================
\begin{flushleft}
\fbox{\textbf{\href{https://www.btc-banco.com/}{GRUPO BITCOIN BANCO}}}}

\vspace{.25cm}
\textbf{\textit{Analista de Estatísticas Sr.}}

[Curitiba de Mai 2019 até Out 2019]
\end{flushleft}

\begin{itemize}
\item Construção dos indicadores de performance (produtividade) da área de atendimento ao cliente, possibilitando identificar a falha no mapeamento de processos de fluxos informacionais internos e ganhando em eficiência e velocidade a resposta aos clientes, subáreas e alta gestão;  
\item Criação de uma cultura de sugestão de planos de ação através de dados e fatos para implantação de sistema de metas definida com focal e gestor de cada área para alta gestão.
\item Indicador de risco alinhado com o \textit{churn} para capturar a “corrida bancária” de nossos clientes frente a psicologia e nervosismo de mercado. 
\item Implantação de processos de fluxos de dados respeitando a LGPD e Seg. Informação de modo organizado e dinâmico;
\item \textit{Dashboards} de dados com planos de ação, simuladores e cenários projetados (via árvore de decisão) para a alta gestão com disparo automático via área para GBB.
\end{itemize}
%===================================================

\begin{flushleft}
\fbox{\textbf{{\href{http://www.fiepr.org.br/para-empresas/estudos-economicos/}{FEDERAÇÃO DAS INDÚSTRIAS DO ESTADO DO PARANÁ}}}}

\vspace{.25cm}
\textit{\textbf{Economista \& Analista Técnico Pleno}}

[Curitiba de Out 2012 até Nov 2018]
\end{flushleft}

\begin{itemize}
\item Melhoria considerável dos modelos de apresentação de dados e estudos com Tableau Public para facilitar as negociações coletivas salariais e estudos de cadeias produtivas prioritárias;
\item Ganhos substanciais com automação de procedimentos de coleta, montagem e armazenamento de bancos de dados com as ferramentas de R+DAX e com produção de relatórios técnicos com Sweave.
\end{itemize}
%===================================================

\newpage

%===================================================
.
\begin{center}
\textbf{Formação Acadêmica}
\end{center}

\rule[6cm]{6in}{0.1pt} 
\vspace{-5.5cm}
%====================================================

\hspace{-.45cm}\textbf{MESTRE EM DESENVOLVIMENTO ECONÔMICO}

\vspace{.25cm}
\textbf{Universidade Federal do Paraná}

De 2008 a 2011

\vspace{.25cm}
\begin{flushleft}
Conclui o programa de mestrado como bolsista do CNPq e FIEP, com uma dissertação que versava sobre o projeto “ID-MPE” que desenvolvi como consultoria aos SEBRAEs.

O link para a dissertação encontra-se  \href{https://www.acervodigital.ufpr.br/handle/1884/25651}{aqui}
\end{flushleft}
\vspace{.25cm}

\hspace{-.45cm}\textbf{BACHAREL EM CIÊNCIAS ECONÔMICAS}

\vspace{.25cm}
\textbf{Universidade Federal do Paraná}

De 2002 a 2007

\vspace{.25cm}

\hspace{-.47cm}Sou economista pela UFPR e como marco acadêmico, até hoje possuo um projeto de pesquisa sobre modelagem de volatilidade de séries financeiras publicado no wiki do laboratório de estatística e geoinformação (LEG) da Universidade (\href{http://www.leg.ufpr.br/doku.php/projetos:ehlers:volprev}{http://www.leg.ufpr.br/doku.php/projetos:ehlers:volprev})  

\hspace{-.47cm}No meu trabalho de monografia reproduzi um modelo econométrico que avaliava o impacto de notícias na dinâmica de preços em ativos financeiros

\vspace{.25cm}


\hspace{-.45cm}\textbf{ESP. EM ESTRATÉGIAS DE ENSINO E APRENDIZAGEM NA EDUCAÇÃO SUPERIOR}

\vspace{.25cm}

\textbf{Universidade Positivo}

De jan 2009 a dez 2010

\begin{flushleft}
Cursei a especialização que objetivava formar professores com melhores habilidades didáticas na prática letiva, porém não cheguei a concluí-la. 

\end{flushleft}
\vspace{.25cm}

\hspace{-.45cm}\textbf{LICENCIATURA E HABILITAÇÃO PLENA EM MATEMÁTICA}
\vspace{.25cm}

\textbf{Universidade do Estado do Paraná – UNESPAR/FAFI}

De jan 1999 a jan 2002

\begin{flushleft}
Praticamente me formei em licenciatura em matemática no campus de União da Vitória – PR, porém decidi me mudar para Curitiba para cursar Economia na UFPR e tranquei o curso.
\end{flushleft}

\rule[6cm]{6in}{0.1pt} 
\vspace{-6cm}
\begin{center}
\textbf{Certificações}
\end{center}


\vspace{.25cm}
\hspace{-.3cm}\textbf{IBM DATA SCIENCE PROFESSIONAL CERTIFICATE}

\hspace{-.25cm}Disponível em \href{https://www.coursera.org/professional-certificates/ibm-data-science}{coursera.org} com auxiliío financeiro.

\begin{flushleft}
Atualmente estudo Ciência de Dados e estou em processo de Certificação Profissional para Cientista de Dados da IBM
\end{flushleft}

\begin{center}
\textbf{Certificação de Idioma}
\end{center}

Proficiência em Língua Estrangeira (Inglês), UFPR (prova de titulação para mestrado)








%===================================================




\newpage
%===================================================
\begin{tikzpicture}[mindmap, grow cyclic, every node/.style=concept, concept color=orange!40, 
	level 1/.append style={level distance=5cm,sibling angle=90},
	level 2/.append style={level distance=3cm,sibling angle=45},]
	\node{Áreas de Interesse}
child { node {Econometria}
	child { node {Análise Fatorial, Multivariada}}
%	child { node {Análise de Séries Temporais}}
	child { node {Modelos de Predição e Previsão (Machine Learning)}}
%	child { node {Modelos de Probabilidade (Probit, Tobit e Logit)}}
	child { node {Modelos de Probabilidade (Probit, Tobit e Logit)}}
%	child { node {Tables and Matrices}}
	child { node {Análise de Séries Temporais}}
}
child { node {Bancos de Dados}
	child { node {Fluxo de informações}}
	child { node {Extração}}
	child { node {Modelagem de Banco de Dados}}
	child { node {Estatísticas Descritivas e Análise Preliminar}}
%	child { node {Title Page}}
}
child { node {Pesquisa de Mercado}
	child { node {Micro econometria}}
	child { node {Estudos de Mercado \textit{data driven}}}
	child { node {Avaliação de cenários de competividade}}
	child { node {API Google Analitycs para e-commerces}}
%	child { node {Themes and Handouts}}
}
child { node {Finanças e Tecnologia}
	child { node {Rastr. via blockchain}}
	child { node {Estratégia de investimentos em criptoativos}}
	child { node {Otimização de Portfólio, \textit{forecast} de preços}}
	child { node {Análise de Viabilidade Econômico Financeira}}
	child { node {Auditoria com Data Science}}
};
\end{tikzpicture}

	


%=============================================================	
\end{document}
%=============================================================	
